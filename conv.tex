\documentclass[oneside]{book}
\usepackage{braket}
\usepackage[latin1]{inputenc}
\usepackage{amsfonts}
\usepackage{amsthm}
\usepackage{amsmath}
\usepackage{mathrsfs}
\usepackage{enumitem}
\usepackage[pdftex]{color,graphicx}
\usepackage{hyperref}
\usepackage{listings}
\usepackage{calligra}
\usepackage{algpseudocode} 
\DeclareFontShape{T1}{calligra}{m}{n}{<->s*[2.2]callig15}{}
\newcommand{\scripty}[1]{\ensuremath{\mathcalligra{#1}}}
\setlength{\oddsidemargin}{0cm}
\setlength{\textwidth}{490pt}
\setlength{\topmargin}{-40pt}
\addtolength{\hoffset}{-0.3cm}
\addtolength{\textheight}{4cm}
\usepackage{amssymb}
\usepackage{graphicx} % Required for the inclusion of images
\setlength\parindent{0pt} % Removes all indentation from paragraphs
\usepackage{float}
\usepackage{makeidx}

\begin{document}

\begin{center}
\textsc{\LARGE Convolución 2D de Gaussianas}\\[0.5cm]
\end{center}

Sea $N(x,\mu,\sigma)$ la función distribución normal con varianza $\sigma^2$ y promedio $\mu$ definida como sigue:

\begin{equation}
N(x,\mu,\sigma) = \frac{1}{\sqrt{2\pi \sigma^2}}e^{- \frac{(x-\mu)^2}{2\sigma^2}}
\end{equation}

Tome las siguientes distribuciones normales bidimensionales:

\begin{equation}
f(x,y) =  N(x,\mu_{x_1},\sigma_{x_1}) N(y,\mu_{y_1},\sigma_{y_1})
\end{equation}

\begin{equation}
g(x,y) =  N(x,\mu_{x_2},\sigma_{x_2}) N(y,\mu_{y_2},\sigma_{y_2})
\end{equation}


El producto de convolución bidimensional se define como:

\begin{equation}
f*g(x,y)  = \int_{\mathbb{R}} \int_{\mathbb{R}} dt d\tau\;  f(x-t,y-\tau) g(t,\tau)
\end{equation}

Como las funciones se pueden separar, se obtiene lo siguiente:

\begin{multline}
\int_{\mathbb{R}} \int_{\mathbb{R}} dt d\tau\;  f(x-t,y-\tau) g(t,\tau)  =\\ \left( \int_{\mathbb{R}} dt\;   N(x-t,\mu_{x_1},\sigma_{x_1})  N(t,\mu_{x_2},\sigma_{x_2})  \right) \times \left( \int_{\mathbb{R}} dt\;   N(y-t,\mu_{y_1},\sigma_{y_1})  N(t,\mu_{y_2},\sigma_{y_2})  \right) 
\end{multline}

Es decir, se obtiene la multiplicación de dos convoluciones en una dimensión. La convolución ordinaria de dos distribuciones normales es:

\begin{equation}
\int_{\mathbb{R}} dt\;   N(x-t,\mu_{x_1},\sigma_{x_1})  N(t,\mu_{x_2},\sigma_{x_2})  =   N\left(x, \mu_{x_1} +\mu_{x_2}, \sqrt{\sigma_{x_1}^2 + \sigma_{x_2}^2}\right)
\end{equation}

Por lo que, se tiene que:

\begin{equation}
f*g(x,y)   =  N\left(x, \mu_{x_1} +\mu_{x_2}, \sqrt{\sigma_{x_1}^2 + \sigma_{x_2}^2}\right) N\left(x, \mu_{y_1} +\mu_{y_2}, \sqrt{\sigma_{y_1}^2 + \sigma_{y_2}^2}\right)
\end{equation}

Ahora con esto se resuelve el siguiente problema. Suponga tiene unas funciones $g$ y $f$ como se definieron anteriormente. Usted desea encontrar una función $K$ tal que pase lo siguiente:

\begin{equation}
f*K(x,y)   =  g(x,y)
\end{equation}

La función $K$ es:

\begin{equation}
K(x,y)   =  N\left(x, \mu_{x_2} - \mu_{x_1} , \sqrt{ \sigma_{x_2}^2 - \sigma_{x_1}^2}\right) N\left(x,  \mu_{y_2}-\mu_{y_1}, \sqrt{  \sigma_{y_2}^2  - \sigma_{y_1}^2}\right)
\end{equation}

\end{document}
